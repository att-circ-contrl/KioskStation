% Kiosk manual - Installation and setup.
% Written by Christopher Thomas.

\chapter{Kiosk Setup}
\label{sect-setup}

\section{Checklist}
\label{sect-setup-checklist}

First-time kiosk setup involves assembling fittings on the kiosk housing 
and installing and cabling electronic equipment inside the electronics bay.
A summary of the steps involved is shown below; photographs and additional
notes are given in Section \ref{sect-setup-howto}.
A cabling diagram for the kiosk is shown in Figure \ref{fig-kiosk-cabling}.

Initial mechanical work:
\begin{itemize}
\item Separate the kiosk face from the electronics bay.
\item With the kiosk face:
\begin{itemize}
\item Remove the outer viewport window from ports that will have cameras
installed.
\item Remove both viewport windows from the port that will have the
cleaning hatch installed.
\item Remove the pump bracket from the side which has the cleaning port.
\item Install the cleaning hatch in its port.
\item Install the pump mounting plate on the remaining pump bracket. The pump
is away from the face of the kiosk, towards the direction of the electronics
bay.
\item Install light filter plates on the viewport windows that will have
cameras. The synchronization LEDs should be close to the camera mounting
arms.
\end{itemize}
\item With the electronics bay:
\begin{itemize}
\item Remove internal shelves.
\item Remove ventilation grilles.
\item Remove fan dummy panels from either the top level or both levels
(for mounting two or four fans, respectively).
\item Remove the external wiring panel from the side nearest the pump.
\item Remove the monitor tray.
\item Seat the touch screen in the monitor tray. Annotate the back of the
monitor tray to show connector locations.
\item Install the monitor and tray in the electronics bay.
\item Install the fans and fan guards.
\item Line the ventilation grilles with HVAC tape to render them glove-safe.
\item Cut filter pads and install the ventilation grilles with filters in
the electronics bay.
\end{itemize}
\end{itemize}

Electronics work:
\begin{itemize}
\item \textbf{NOTE} - This does not include eye-tracker installation.
Power cabling is placed in the eye-tracker bay.
\item \textbf{NOTE} - During each cabling step, excess cable length should
be coiled and zip-tied, and routed cables should be zip-tied to cable mounts
where appropriate.
\item Install power distribution bars at the bottom of the bay (turned off).
\item Reinstall bay shelves.
\item Install fan power plug and connect fans.
\item Install touch screen power supply. Connect power, display, and USB
cables to the touch screen.
\item Install small monitor, keyboard, and mouse on the right-hand side of
the middle shelf. Install monitor power supply in the bottom of the bay.
Route data cables to the top shelf.
\item Install router with two ethernet cables connected to LAN ports on the
left-hand side of the middle shelf. Install router power supply in the 
bottom of the bay. Route ethernet cables to the top shelf. Ensure that the
router's power switch is turned on, if it has one.
\item Install Neurarduino on the top shelf, left-hand side.
\item Install pump reward box on the top shelf, left-hand side. Cable the
pump reward box to the Neurarduino.
\item Install pump reward fob. Use hook-and-loop cable ties to hang it from
the electronics bay carrying handle on the pump side of the unit. Cable the
fob to the pump reward box.
\item Install the Unity machine's power supply in the bottom of the bay.
Route it to the left-hand side of the top shelf.
\item Install Unity machine on top of the Neurarduino. Attach Unity machine
cable connections (which should already be routed to the top shelf).
\item Install LED box on the top shelf, right-hand side. Route BNC cables 
from the LED box out appropriate camera cabling ports; do not coil the 
LED BNC cables yet.
\item Install the NeuroCam machine's power supply in the bottom of the bay.
Route it to the right-hand side of the top shelf.
\item Install NeuroCam and NeuroCam USB hub. Cable the LED box to the
USB hub and the USB hub to the NeuroCam. Attach the NeuroCam power and
ethernet cables.
\end{itemize}

Electronics tests:
\begin{itemize}
\item Turn on power bar. Press power switches on Unity and NeuroCam
computers. Wait 60 seconds for machines to boot.
\item Check that the Unity machine's login display is shown on the touch 
screen and on the small monitor.
\item Check that an authorized machine can connect to the NeuroCam via the
kiosk's router.
\item Perform graceful shutdowns of the Unity machine and NeuroCam machine.
\item Turn off power bar.
\end{itemize}

Final mechanical work:
\begin{itemize}
\item Reattach the kiosk face to the electronics bay.
\item Install cameras.
\item Screw pump control cable on to pump. Use pliers for this.
\item Install pump.
\item Connect LED BNC cables to camera port LEDs. Coil and zip-tie excess 
cabling \textit{inside} the electronics bay. Do \textbf{not} zip-tie BNC
cables to the kiosk face exterior frame unless this is unavoidable. If it
is unavoidable, keep the zip-ties loose enough that they can be cut without
damaging the BNC cables.
\item Connect camera USB cables to the NeuroCam computer. Coil excess
cabling \textit{inside} the electronics bay. Zip-tie the coils, but do
\textbf{not} zip-tie the coils or cables to cable mounts (it must be possible
to disconnect and withdraw the camera USB cables). Exterior camera cables
may be zip-tied to the kiosk face's frame.
\item Connect the pump control cable to the pump box. Coil excess cabling
\textit{inside} the electronics bay. Zip-tie the coils, but do \textbf{not}
zip-tie the coils or cables to cable mounts (it must be possible to
disconnect and withdraw the pump control cable).
\item Zip-tie cables that pass through the lower exterior cable entry port
to the adjacent handle. Keep this loose enough that it can be cut without
damaging the cables.
\item Replace camera cover.
\item Install pump hoses.
\item Mount the kiosk so that the underside is accessible.
\item Install sipper tube.
\item Make remaining pump hose connections.
\end{itemize}

\begin{figure}[h]
\includegraphics[width=0.99\columnwidth]{cabling/kiosk-cabling-2019.pdf}
\caption{Kiosk cabling daigram.}
\label{fig-kiosk-cabling}
\end{figure}

\clearpage
\section{Kiosk Assembly Notes}
\label{sect-setup-howto}

%
%
\subsection{Viewports and Hatch}
\label{sect-setup-howto-viewports}

\begin{tabular}{ccc}
\includegraphics[width=0.4\columnwidth]
{photos/install-20181106/face-before.jpg} &
~~~ &
\includegraphics[width=0.4\columnwidth]
{photos/install-20181106/hatch.jpg} \\
\end{tabular}

%
%
\clearpage
\subsection{Light Filters and Cameras}
\label{sect-setup-howto-lightfilt}

The LEDs should be as close as possible to the camera mounts. The cameras
must have LEDs in their field of view when in their final positions.

\begin{tabular}{ccc}
\includegraphics[height=0.55\columnwidth]
{photos/install-20181106/lightfilts.jpg} &
~~~ &
\includegraphics[height=0.55\columnwidth]
{photos/install-20181106/cameras.jpg} \\
\end{tabular}

%
%
\clearpage
\subsection{Electronics Bay Prep}
\label{sect-setup-howto-bay}

\begin{tabular}{ccc}
\includegraphics[width=0.4\columnwidth]
{photos/install-20181106/bay-before.jpg} &
~~~ &
\includegraphics[width=0.4\columnwidth]
{photos/install-20181106/bay-stripped.jpg} \\
 & ~ & \\
\multicolumn{3}{c}{%
\includegraphics[width=0.85\columnwidth]
{photos/install-20181106/bay-parts.jpg}%
} \\
\end{tabular}

%
%
\clearpage
\subsection{Touch-Screen Tray}
\label{sect-setup-howto-touch}

\begin{tabular}{ccc}
\includegraphics[height=0.45\columnwidth]
{photos/install-20181106/ts-installed.jpg} &
~~~ &
\includegraphics[height=0.45\columnwidth]
{photos/install-20181106/ts-labels.jpg} \\
\end{tabular}

%
%
\clearpage
\subsection{Fans}
\label{sect-setup-howto-fans}

The photographed version of the kiosk needed adapter plates to mount the
fans. Later versions mount the fans directly to the electronics cabinet.

\textbf{NOTE} - Fans on one side blow into the electronics bay (intake fans),
and fans on the other side blow out of the electronics bay (exhaust fans).
One pair of fans has the stickers facing in and one pair has the stickers
facing out.

\begin{center}
\begin{tabular}{ccc}
\includegraphics[height=0.3\columnwidth]
{photos/install-20181106/fan-parts.jpg} &
~~~ &
\includegraphics[height=0.3\columnwidth]
{photos/install-20181106/fan-screws.jpg} \\
\end{tabular}

\begin{tabular}{ccc}
\includegraphics[height=0.6\columnwidth]
{photos/install-20181106/fan-assembled.jpg} &
~~~ &
\includegraphics[height=0.6\columnwidth]
{photos/install-20181106/fan-installed.jpg} \\
\end{tabular}
\end{center}

%
%
\clearpage
\subsection{Air Filters}
\label{sect-setup-howto-airfilt}

\begin{tabular}{ccc}
\multicolumn{3}{c}{%
\includegraphics[height=0.5\columnwidth]
{photos/install-20181106/air-taped.jpg}%
} \\
 & ~ & \\
\includegraphics[height=0.5\columnwidth]
{photos/install-20181106/air-batting.jpg} &
~~~ &
\includegraphics[height=0.5\columnwidth]
{photos/install-20181106/air-installed.jpg} \\
\end{tabular}

%
%
\clearpage
\subsection{Power Bars and Shelves}
\label{sect-setup-howto-power}

\textbf{NOTE} - The eye-tracker is not installed in the photographed version
of the kiosk. Power cabling is placed in the eye-tracker bay.

\begin{center}
\includegraphics[width=0.9\columnwidth]
{photos/install-20181106/power.jpg}
\end{center}

%
%
\clearpage
\subsection{Small Monitor, Keyboard, and Mouse}
\label{sect-setup-howto-smallmon}

\textbf{NOTE} - The setup photographed did not have a mouse at the time of
installation. A keyboard with integrated touchpad or other pointing device
may be preferable.

\begin{center}
\includegraphics[width=0.9\columnwidth]
{photos/install-20181106/small-mon.jpg}
\end{center}

%
%
\clearpage
\subsection{Wireless Gateway}
\label{sect-setup-howto-router}

\textbf{NOTE} - The antennae for the wireless gateway should be far away
from the metal walls, and nearby cabling should never run parallel to the
antennae.

\begin{center}
\includegraphics[width=0.9\columnwidth]
{photos/install-20181106/small-mon-gateway.jpg}
\end{center}

%
%
\clearpage
\subsection{Neurarduino and Pump Control Box}
\label{sect-setup-howto-ard}

\begin{center}
\includegraphics[width=0.9\columnwidth]
{photos/install-20181106/ard-pbox-wiring.jpg}
\end{center}

%
%
\subsection{Unity Machine}
\label{sect-setup-howto-unity}

\begin{center}
\begin{tabular}{c}
\includegraphics[width=0.6\columnwidth]
{photos/install-20181106/unity-wiring.jpg} \\
\includegraphics[width=0.8\columnwidth]
{photos/install-20181106/ard-unity-placement.jpg} \\
\end{tabular}
\end{center}

%
%
\clearpage
\subsection{LED Box}
\label{sect-setup-howto-ledbox}

\begin{center}
\includegraphics[width=0.9\columnwidth]
{photos/install-20181106/led-box.jpg}
\end{center}

%
%
\subsection{NeuroCam Machine}
\label{sect-setup-howto-neurocam}

\begin{center}
\includegraphics[width=0.9\columnwidth]
{photos/install-20181106/led-ncam-placement.jpg}
\end{center}

%
%
\clearpage
\subsection{External Cabling and Pump}
\label{sect-setup-howto-extcable}
\label{sect-setup-howto-pump}

\begin{tabular}{ccc}
\includegraphics[height=0.5\columnwidth]
{photos/install-20181106/ext-cabling.jpg} &
~~~ &
\includegraphics[height=0.5\columnwidth]
{photos/install-20181106/pump.jpg} \\
\end{tabular}

%
%
\subsection{Camera Cover}
\label{sect-setup-howto-cover}

\begin{tabular}{ccc}
\includegraphics[height=0.47\columnwidth]
{photos/install-20181106/cover-front.jpg} &
~~~ &
\includegraphics[height=0.47\columnwidth]
{photos/install-20181106/cover-back.jpg} \\
\end{tabular}

%
%
\clearpage
\subsection{Pump Hoses}
\label{sect-setup-howto-hoses}

\fixme{Need photos for this.}

%
% This is the end of the file.
