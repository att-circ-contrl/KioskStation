% Kiosk manual - Routine activities.
% Written by Christopher Thomas.

\chapter{Routine Activities}
\label{sect-routine}

\section{Startup and Shutdown}
\label{sect-routine-startupdown}

To power on the electronics bay equipment:

\begin{itemize}
%
\item Turn on the leftmost power bar, and check that the rightmost one
(plugged into the left bar) is also turned on.
%
\item Check that the fans have spun up. They should turn on within a few 
seconds of receiving power.
%
\item Check that the wireless gateway is on (there should be visible lights 
blinking).

If there are no lights, check that the gateway's power button is 
turned on (if it has one). This is the \textbf{large, exposed} button - do 
not press any small or inset buttons (those either reset the gateway to 
default settings or tell it to use an insecure reconfiguration method - 
both are bad).
%
\item Press and hold the NeuroCam machine's power button until it lights up
and the machine's fan spins up.

This should only take about a second. Do not hold it for longer than 3
seconds - holding it 4 seconds or longer forces an immediate shutdown, which
can cause corruption.
%
\item Press and hold the Unity machine's power button until it lights up and
the machine's fan spins up.

Per previous entry, this should only take about a second.
%
\item Check that the small monitor and the touch screen are both showing
a Windows boot or login screen.

If either one is not showing a screen, check that it's powered on.
%
\item Check that you can access the NeuroCam gateway, and that you can
access the NeuroCam's control web page.

The NeuroCam takes 30-60 seconds to boot, so be sure to leave sufficient
time before accessing it.
%
\item Once everything's running properly, close and lock the electronics 
bay doors.
%
\end{itemize}

\clearpage
To safely shut down the electronics bay equipment:

\begin{itemize}
%
\item Log in to the NeuroCam's web page over the wireless network. Click
the ``shutdown'' button.
%
\item Check that the NeuroCam's power light turns off. This may take
several seconds.

\begin{itemize}
\item If the NeuroCam does not power off, it can be forced to power down by
pressing and holding the power button for 4-5 seconds.

\textbf{NOTE} - This may cause data corruption and other problems, so try to
avoid this if possible.
\end{itemize}
%
\item If the Unity machine is logged in:
\begin{itemize}
\item \fixme{key sequence for power-off goes here}.
\end{itemize}
%
\item If the Unity machine is not logged in:
\begin{itemize}
\item Press a harmless key (such as ``shift'') to wake up the display.
\item Press ``tab'' until the power icon is highlighted (this will take 
multiple keystrokes).
\item Hit ``enter'' to expand the power menu.
\item Use the up and down cursor keys to highlight ``Shut Down''.
\item Hit ``enter''.
\end{itemize}
%
\item Check that the Unity machine's power light turns off. This may take
several seconds.

\begin{itemize}
\item If the Unity machine does not power off, it can be forced to shut down
per above. This may cause data corruption and other problems, as noted.
\end{itemize}
%
\item Locate the power switch on the left power bar (the one whose cord
goes outside), and turn it off.
%
\item Once everything's shut down, close and lock the electronics bay doors.
%
\end{itemize}

%
% This is the end of the file.
